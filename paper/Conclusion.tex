\section{Conclusion}\label{sec:conclusion}

For years, many of the leading industry memory models have been so complicated to understand and to analyze that the status quo was simply to live with an incomplete and underspecified memory model.
Academics would attempt to build axiomatic and operational models and then to prove them equivalent, but these models and proofs were subject to frequent breakage and refinement due to the thorniness of the issues at hand.
Other models were simply never updated to modern standards, and were therefore left with definitions fence ordering, same-address ordering, and/or dependency ordering that are today well known to be insufficient.
This has led to no shortage of confusion in the broader understanding of memory models in the field.

In response to the recently emerging trend back towards atomic memory models, we present GAM, a flexible operational \emph{and} axiomatic memory model definition that is \emph{parameterized} by the set of fences in the model.
GAM corrects the preserved program order definition oversights present in memory models from past generations, and it reduces the definition of fence behavior into localized intra-thread ordering specifications that can be easily understood in isolation.
GAM also comes with proofs of equivalence between its axiomatic and operational models, thereby overcoming the obstacle that many previous memory models have faced in being far too complicated to understand or to work with.
The equivalence makes it much easier for architects, programmers, and theoreticians to each simply use the variant that they find easiest to work with.

Finally, GAM also makes it easy to understand the implications of tweaking a memory model's definition.
It is easy to add new fences that trade off strength for performance, for example.
It is also possible to remove behaviors; as we show, forbidding load-store reordering altogether allows GAM to be reduced to an even simpler I2E-based definition.
We believe that all of these features will go a long way towards eliminating the worst of the subtleties and corner cases that have most of the memory models of past generations.

