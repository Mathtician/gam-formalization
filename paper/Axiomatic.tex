

\section{COM: an Alternative Axiomatic Model}\label{sec:COM}

In this section, we present an alternative (but still parameterized) axiomatic formulation that is perhaps less intuitive, but nevertheless in common use due to its computational efficiency.
We call this formulation the \emph{COM} model (where ``COM'' stands for communication, as described below).
We first present a proof of equivalence between the GAM axioms and the COM axioms.  This in turn implies that COM is also equivalent to the operational definition of GAM.  We then implement both axiomatic models in Alloy~\cite{alloy} in order to perform sanity checking and empirical testing of the models and of the proofs.

\subsection{The COM Axioms}
The COM model is defined in terms of three basic relations and three derived relations, plus $\PreservePO$:
\begin{itemize}
  \item Basic relations:
    \begin{itemize}
      \item Program order ($\ProgOrd$), as before
      \item Reads-from ($\ReadFrom$), as before
      \item Coherence ($\Coh$), a total order over the writes to each memory address
    \end{itemize}
  \item Derived relations:
    \begin{itemize}
      %\item Reads-from internal ($\Rfi$), which is the subset of $\ReadFrom$ for which both the read and the write are in the same thread
      \item Reads-from external ($\Rfe$), which is the subset of $\ReadFrom$ for which both the read and the write are in different threads
      \item From-reads ($\Fr$=${\RfInv};\Coh$), which relates each read $r$ to every write which follows the $\ReadFrom$-source of $r$ in $\Coh$.  ($\RfInv$ indicates the inverse of $\ReadFrom$)
      \item Program order, same location ($\PoLoc$), which is the subset of program order that relates memory accesses to the same memory address
    \end{itemize}
  %\item Parameterized relations:
  %  \begin{itemize}
  %    \item Preserved program order ($\PreservePO$), as before.  We assume that $\PreservePO$ contains at least the edges of Definition \ref{def:ppo-same-addr}.
  %  \end{itemize}
\end{itemize}

Another derived relation $<_{com}=\ReadFrom \cup \Coh \cup \Fr$ is often defined as a convenient shorthand in this style of model (hence our choice of the name ``COM''), but we do not use it in this paper.

\noindent In the COM formulation, an execution is legal if it satisfies the following two axioms:

\begin{itemize}
  \item \textbf{Axiom SC-per-Location:} $\acyclic(\ReadFrom \cup \Coh \cup \Fr \cup \PoLoc)$
  \item \textbf{Axiom Causality:} $\acyclic(\Rfe \cup \Coh \cup \Fr \cup \PreservePO)$
\end{itemize}

\subsection{Equivalence of GAM and COM}

The complete proofs are provided in Appendix~\ref{sec:gamcom}.  We provide an intuition here.

To prove that GAM $\subseteq$ COM, we must do two things: 1) find a suitable choice of $\Coh$, which does not exist in the GAM model, and 2) prove that if the GAM axioms are satisfied, the COM axioms are satisfied.  Of course, the natural choice for $\Coh$ is to simply take the restriction of $\MemOrd$ that relates only stores to the same address, and that is indeed what we use.  It remains to show that for any choice of $\MemOrd$ in the GAM axioms, the two COM axioms are satisfied.

We start with a lemma:
\begin{lemma}\label{lem:com_in_memord}
  All of $\Rfe$, $\Coh$, $\Fr$, and $\PreservePO$ are contained in $\MemOrd$.
\end{lemma}
\begin{proof}
  Straightforward; see appendix.
\end{proof}

With this lemma, it is easy to show that the Causality axiom is satisfied:
\begin{theorem}
  The Causality axiom is satisfied.
\end{theorem}
\begin{proof}
  By Lemma~\ref{lem:com_in_memord}, the union $\Rfe \cup \Coh \cup \Fr \cup \PreservePO$ is a subset of $\MemOrd$.  Therefore, since $\MemOrd$ is acyclic, $\Rfe \cup \Coh \cup \Fr \cup \PreservePO$ must also be acyclic.
\end{proof}

The SC-per-Location axiom will take a bit more work to prove.
To start, define $\Eco$ as the union of the following relations:
\begin{itemize}
  \item $\Coh$ (Write to Write)
  \item $\Fr$ (Read to Write)
  \item ${\Coh}^*;\ReadFrom$ (Write to Read)
  \item $\RfInv;{\Coh}^*;\ReadFrom$ (Read to Read)
\end{itemize}

\begin{lemma}\label{lem:eco_either}
  For all pairs $i_1$, $i_2$ of memory accesses to the same address, either $i_1\Eco i_2$ or $i_2\Eco i_1$.
\end{lemma}
\begin{proof}
  By construction; see appendix.
\end{proof}

If $i_1$ and $i_2$ are related in program order, then the $\Eco$ direction must match:
\begin{lemma}\label{lem:eco_poloc}
  If $i_1\PoLoc i_2$, then $i_1\Eco i_2$.
\end{lemma}
\begin{proof}
  The alternative of $i_2\Eco i_1$ results in a contradiction, except for one case where it overlaps $i_1\Eco i_2$.  See appendix.
\end{proof}

\begin{theorem}
  The SC-per-Location axiom is satisfied.
\end{theorem}
\begin{proof}
  (abbreviated; see appendix)

  First, by Lemma~\ref{lem:eco_poloc}, all $\PoLoc$ edges involving at least one write can be converted into sequences containing only $\ReadFrom$, $\Coh$, and $\Fr$.  So we consider only cycles with $\ReadFrom$, $\Coh$, $\Fr$, and read-to-read $\PoLoc$ edges.  Replace every instance of read-read $\PoLoc$ in the cycle with $\RfInv;{\Coh}^*;\ReadFrom$ per Lemma~\ref{lem:eco_poloc}.  Now, because $\Coh$ and $\Fr$ both target writes, every appearance of $\RfInv$ must be preceded either by $\ReadFrom$ or by $\RfInv;{\Coh}^*;\ReadFrom$.  In particular, every appearance of $\RfInv$ must be preceded directly by $\ReadFrom$.  Since $\ReadFrom;\RfInv$ is the identity function, all appearances of $\RfInv$ in the cycle can be eliminated by simply removing each $\ReadFrom;\RfInv$ pair in the cycle.  This leaves a cycle with only $\ReadFrom$, $\Coh$, and $\Fr$, which is a contradiction.
\end{proof}

\subsection{COM $\subseteq$ GAM}

This direction is easier.  Given $\ProgOrd$, $\ReadFrom$, and $\Coh$, we must find a suitable $\MemOrd$.
By the Causality axiom, $\Rfe \cup \Coh \cup \Fr \cup \PreservePO$ is acyclic, and hence there is at least one total ordering compatible with it.  We show that any such total ordering satisfies GAM.
The Inst-Order axiom is true by construction, and hence we must only show that the Load-Value axiom is satisfied.

\begin{theorem}
  Any $\MemOrd$ which is a total ordering of $\Rfe \cup \Coh \cup \Fr \cup \PreservePO$ satisfies the Load-Value axiom.
\end{theorem}
\begin{proof}
  If $w\ReadFrom r$, then either $w\Rfi r$ or $w\Rfe r$.  In the first case, $w\ProgOrd r$, or else it would contradict the SC-per-Location axiom.  In the second case, $w\MemOrd r$ by construction of $\MemOrd$.  In either case, $w$ must be in the candidate set
  \[ \{
    \mathsf{St}\ a\ v'\ |\ \mathsf{St}\ a\ v' \ProgOrd
    \mathsf{Ld}\ a\ \vee\ \mathsf{St}\ a\ v' \MemOrd \mathsf{Ld}\ a \}.
  \]
It remains to be shown that $w$ is in fact the $\MemOrd$-maximal element of that candidate set.

  Suppose that $w$ is not maximal.  Then there is some other write $w'$ to the same address $a$ such that $w\MemOrd w'$ and either $w'\ProgOrd r$ or $w'\MemOrd r$.  But then by definition, $r\Fr w'$, and $\Fr$ cannot contradict either $\ProgOrd$ (by SC-per-Location) or $\MemOrd$ (by construction of $\MemOrd$).  Hence we have a contradiction.
\end{proof}


\subsection{Empirical Validation}

We also used model checking to confirm the validity of the proof of equivalence between GAM and COM.
We encoded both models into Alloy~\cite{alloy,memalloy}, a relational model finder backed by a SAT solver, and checked for any mismatches.
The definition of this model is shown in Appendix~\ref{sec:alloy}.
In keeping with the spirit of the proofs, $\PreservePO$ is entirely parameterized; there is no explicit notion of fence or dependency in this version of the model.  We only assume that Definition~\ref{def:ppo-same-addr} always holds.
Under these conditions, Alloy verifies in roughly one hour that no counterexamples are found for tests with up to seven instructions.
